\chapter{软件质量特性}


详细说明项目任何其他的质量特性。该特性对客户和开发者都非常重要。考虑的方面包括:适应性,可用性,正确性,灵活性,交互工作能力,可维护性,可移植性,可靠性,可重用性,鲁棒性,可测试性和可用性等。定量的详细描述这些特性,尽可能的可验证。对不同属性之间的重要性加以阐述,如:易用性比易学性更重要。



\section{适应性}
考虑到Android平台本身的共通性,通讯系统能正常的运行在绝大部分的Android5.0.1及以上版本的机器上。至于windows平台,则支持xp及以上的操作系统,需要包含微软提供的.Net4.0的组件即可。


\section{交互工作能力}
通讯系统有着完整的用户友好的UI,易于上手。

\section{灵活性}
通讯系统既可以和好友一对一交流,又可以在群组多对多交流,还可以随机配对,和大厅里的任何一个用户交流。

\section{可维护性}
由于用户端和服务器端的分离,用户端的升级只需要通过软件提供商更新安装包就可以了。而后台服务器端则是至少24小时备份一次数据库,至少每周进行一次人工维护。并且所有维护均会留下维护记录,所有的错误信息也会留下log,可维护性得以保障。

\section{可移植性}
通讯系统由Java开发,我们不仅准备了Android和Windows的两种版本,也使得打包封装和移植到别的系统简单。

\section{可测试性}
通讯系统提供了专门用于测试的接口,并且还会提供开发中的各种中间版本作为升级记录。

\section{安全可靠性}
通讯系统在http,tcp等网络传输协议之外,发送的报文还会进行一层RSA加密,大大增强信息的安全性。此外对于绑定邮箱的用户还会有额外的异常登录验证,确保用户账户的安全性。并且开发人员绝不会收集用户的信息和聊天记录,通讯系统是一个绝对的绿色安全可靠的软件。

