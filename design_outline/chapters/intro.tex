\chapter{引言}
\section{编写目的}
在本项目的前一阶段,也就是需求分析阶段,已经将系统用户对本系统的需求做了详细的阐述,这些用户需求已经在上一阶段中对不同用户所提出的不同功能,实现的各种效果做了调研工作,并在需求规格说明书中得到详尽得叙述及阐明。

本阶段已在系统的需求分析的基础上,对即时聊天工具做概要设计。主要解决了实现该系统需求的程序模块设计问题。包括如何把该系统划分成若干个模块、决定各个模块之间的接口、模块之间传递的信息,以及数据结构、模块结构的设计等。在以下的概要设计报告中将对在本阶段中对系统所做的所有概要设计进行详细的说明,在设计过程中起到了提纲挈领的作用。

在下一阶段的详细设计中,程序设计员可参考此概要设计报告,在概要设计即时聊天工具所做的模块结构设计的基础上,对系统进行详细设计。在以后的软件测试以及软件维护阶段也可参考此说明书,以便于了解在概要设计过程中所完成的各模块设计结构,或在修改时找出在本阶段设计的不足或错误。

本文档主要由开发人员撰写,意在指明即时通讯软件系统(以下
简称通讯系统)的具体开发明细,记录下通信软件。
本文档还可作为软件开发过程中的备案,为后续需求的提出给出参考。此
外,非软件开发者可以将这个文件视为我们的即时通讯系统的开发纲要

\section{项目背景}
随着网络通信和信息技术的发展,现代社会人与人之间的交流越来越重要,同时也越来越便捷,并且出现了如Facebook、QQ等著名社交软件。于是我们组决定立足于当下标准,设计一款同样有着及时,便捷,潮流的通讯软件。

\section{术语}
[列出本文档中所用到的专门术语的定义和外文缩写的原词组]
\begin{table}[htbp]
\centering
\caption{术语表} \label{tab:terminology}
\begin{tabular}{|c|c|}
    \hline
    缩写、术语 & 解释 \\
    \hline
    c & d \\
    \hline
\end{tabular}
% \note{这里是表的注释}
\end{table}