\chapter{数据结构设计}
\section{逻辑结构设计}
\subsection{用户管理系统数据结构设计}
用户信息
	\begin{itemize}  	
		\item 用户名 varchar(48)
		\item 密码盐 char(10)
		\item MD5值 char(32)
	\end{itemize}
用户关系信息
	\begin{itemize}  	
		\item 用户名1 varchar(48)
		\item 用户名2 varchar(48)
\end{itemize}
离线消息信息
	\begin{itemize}
		\item 发送用户 varchar(48)
		\item 接收用户 varchar(48)
		\item 信息     varchar(1024)
		\item 发送时间 datetime
	\end{itemize}	
\subsection{客户端数据结构}
数据包
	\begin{itemize}
		\item 消息类型 MessageType
		\item 字符串列表 List<String>
	\end{itemize}	
数据包序列化结构
	\begin{itemize}
		\item 消息类型 byte[1]
		\item 消息长度 byte[2]
		\item 字符串1  byte[n1]
		\item 分隔符   byte[1] 0x00 
		\item 字符串2  byte[n2]
		\item 分隔符   byte[1] 0x00
		\item 字符串3  byte[n2]
		\item ...
		\item 字符串m  byte[nm]
\end{itemize}	
联系人列表
\begin{itemize}
	\item 列表 List<String>
\end{itemize}
未收消息列表
	\begin{itemize}
		\item 列表 Map<String,List<Packet>> Packet:数据包结构\\
	\end{itemize}
	表示一个值是数据包列表的字典,键是对应的联系人。
\subsection{服务器端数据结构}
	同客户端,并且加上以下:\\
	连接列表 List<Socket> \\
	连接字典 Map<username,Socket>\\
	IP字典  Map<IP,Socket>\\
	连接字典反向 Map<Socket,username>
\section{物理结构设计}
	数据包都转换成字节数组。

\section{数据结构与程序模块的关系}
[此处指的是不同的数据结构分配到哪些模块去实现。可按不同的端拆分此表]
\begin{table}[htbp]
\centering
\caption{数据结构与程序代码的关系表} \label{tab:datastructure-module}
\begin{tabular}{|c|c|c|c|c}
    \hline
    · & TCP接收程序 & 数据包解码程序 & 客户端主程序 & 服务器端主程序\\
    \hline
    数据包字节序列 & Y & . & . & .\\
    \hline
    数据包 & · & Y & Y & Y \\
    \hline
    联系人列表 & · & . & Y & Y\\
    \hline
    离线信息列表 & · & . & Y & Y \\
    \hline
\end{tabular}
\note{各项数据结构的实现与各个程序模块的分配关系}
\end{table}