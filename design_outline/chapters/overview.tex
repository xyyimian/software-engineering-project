\chapter{任务概述}
本系统的目标是实现一个xxx系统,包括客户端、服务器端两个部分。

客户端面向xxx用户,为用户提供xx和xx服务。

\section{目标}
实现xxx系统,实现需求规格说明书中所描述的xx功能、xxx功能和xxx功能,并且保证系统的健壮性和数据安全。

\section{开发与运行环境}

\subsection{开发环境的配置}
\begin{table}[htbp]
\centering
\caption{开发环境的配置} \label{tab:development-environment}
\begin{tabular}{|c|c|c|}
    \hline
    类别 & 标准配置 & 最低配置 \\
    \hline
    计算机硬件 & \tabincell{c}{基于x86结构的CPU\\ 主频>=2.4GHz\\ 内存>=8G\\ 硬盘>=200G} & \tabincell{c}{基于x86结构的CPU\\ 主频>=1.6GHz\\ 内存>=512M\\ 硬盘>=2G} \\
    \hline
    计算机软件 & \tabincell{c}{Linux (kernel version>=4.10)\\ GNU gcc (version>=6.3.1)} & \tabincell{c}{Linux (kernel version>=3.10)\\ GNU gcc (version>=5.4)} \\
    \hline
    网络通信 & \tabincell{c}{至少要有一块可用网卡\\ 能运行IP协议栈即可} & \tabincell{c}{至少要有一块可用网卡\\ 能运行IP协议栈即可} \\
    \hline
    其他 & 采用MySQL数据库 & 采用MySQL数据库 \\
    \hline
\end{tabular}
% \note{这里是表的注释}
\end{table}

\section{需求概述}
功能需求包括:


\section{条件与限制}
本节至少要与需求说明文档中相关章节相一致。
